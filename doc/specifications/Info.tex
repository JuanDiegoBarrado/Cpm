\documentclass[10pt,a4paper]{article}

%% Formateo del título del documento
\title{\Huge C$+$$-$}
\author{Juan Diego Barrado Daganzo, Javier Saras González y Daniel González Arbelo \\ 4º de Carrera}
\date{\today\footnote{Este documento se actualiza, para consultar las últimas versiones entrar en el enlace \url{https://github.com/JuanDiegoBarrado/PracticaPL}}}

%% Formateo del estilo de escritura y de la pagina
\pagestyle{plain}               % Estilo de página
\setlength{\parskip}{0.35cm}    % Edicion de espaciado
\setlength{\parindent}{0cm}     % Edicion de sangría
\clubpenalty=10000              % Llíneas viudas NO
\widowpenalty=10000             % Líneas viudas NO

%% Para establecer las medidas de los margenes
\usepackage[top=2.5cm, bottom=2.5cm, left=3cm, right=3cm]{geometry} 
%% Para que el idioma por defecto sea español
\usepackage[spanish]{babel}
%% Para poder subrayar entornos especiales como las secciones
\usepackage{ulem}

%% Texto matematico y simbolos especiales
\usepackage{amsmath}    % Paquete para mates
\usepackage{amsfonts}   % Paquete para mates
\usepackage{amssymb}    % Paquete para mates
\usepackage{stmaryrd}   % Paquete para mates
\usepackage{latexsym}   % Paquete para mates

%% Paquete para incluir imágenes y ruta de la carpeta de las imágenes
\usepackage{graphicx}
\graphicspath{{./fotos/}}

%% Paquete para tener hipervínculos y referencias cruzadas
\usepackage[colorlinks=true]{hyperref}
\hypersetup{
	urlcolor=red,
	linkcolor=blue,
}

%% Paquete para incluir código con coloreado sintáctico
\usepackage{listings}
\lstdefinelanguage{C+-}
{
  keywords={
    if, else, guail, breic, continue,
    return
    },
  keywordstyle=\color{blue},
  emph={int, bul, func},
  emphstyle=\color{purple},
  commentstyle=\color{green},
  stringstyle=\color{red},
  sensitive=true,
  morecomment=[l]{//},
}

\lstset{
    language=C+-,
    basicstyle=\ttfamily\small,
    keywordstyle=\color{blue},
    commentstyle=\color{green},
    stringstyle=\color{red},
    numbers=left,
    numberstyle=\tiny\color{gray},
    breaklines=true,
    frame=shadowbox,
    rulesepcolor=\color{black},
    backgroundcolor=\color{white},
    tabsize=2,
    gobble=12,
    linewidth=0.65\linewidth,
    float=h
}


%% Definicion de operadores especiales para simplificar la escritura matematica
\DeclareMathOperator{\dom}{dom}
\DeclareMathOperator{\img}{img}
\DeclareMathOperator{\rot}{rot}
\DeclareMathOperator{\divg}{div}
\newcommand{\dif}[1]{\ d#1}

%% Paquete e instrucciones para la generacion de los dibujos
\usepackage{pgfplots}
\pgfplotsset{compat=1.17}
\usepackage{tkz-fct}
\usepackage{pstricks}
\usepackage{pstcol} 
\usepackage{pst-node}
\usepackage{pst-plot}

%% Paquetes extra
\usepackage{centernot}  % Paquete para tachar cosas
\usepackage{appendix}   % Paquete para apéndice
\usepackage{verbatim}   % Paquete para comentar bloques de código de LaTeX

\begin{document}
\maketitle
\tableofcontents

\section{Especificaciones técnicas del lenguaje}
\subsection{Identificadores y ámbitos de definición}
El lenguaje posee las siguientes características:
\begin{itemize}
    \item \textbf{Declaración de variables}: se pueden declarar variables sencillas de los tipos definidos y variables \textit{array} de estos tipos, de cualquier dimensión. Los nombres de los identificadores han de ser expresiones alfanuméricas que no comiencen por números y que posiblemente tengan el caracter ``\_''.
    \item \textbf{Bloques anidados}: se permiten las anidaciones en condicionales, bucles, funciones, etc. Si dos variables tienen el mismo nombre, la más profunda (en la anidación) tapa a la más externa.
    \item \textbf{Funciones}: se permite la creación de funciones y la declaración implícita dentro de otras. El paso por valor y por referencia de cualquier tipo a las funciones está garantizado.
    \item \textbf{Punteros}: para cada tipo se puede declarar un puntero a una variable de ese tipo, mediante la asignación de su dirección de memoria a la variable puntero.
    \item \textbf{Registros y clases}: se incluyen dos tipos adicionales: los registros como ``saco de datos'' ---sin métodos--- y las clases, tanto con datos como con métodos de función.
    \item \textbf{Declaración de constantes}: se incluye la posibilidad de declarar constantes por parte del usuario.
\end{itemize}

\subsection{Tipos}
La declaración de tipos ha de hacerse de manera explícita y de forma previa al lugar donde se emplee el identificador, es decir, que para poder usar una variable tengo que haberla declarado antes.

\subsubsection{Enteros y booleanos}
Los tipos básicos del lenguaje son los enteros y los booleanos. La sintaxis de declaración de estos tipos es la siguiente:
\begin{itemize}
    \item \textbf{Enteros}: \texttt{int var;}
    \item \textbf{Booleanos}: \texttt{bul var;}
\end{itemize}
Entre las operaciones habilitadas para el tipo:
$$FALTA ESTO$$

\subsubsection{Clases y registros}
Como tipos adicionales hemos incluido los registros, las clases y las funciones. La sintaxis de declaración es la siguiente:
\begin{itemize}
    \item \textbf{Clases}: \texttt{clas var \{...\};}
    \item \textbf{Registros}: \texttt{estrut var \{...\};}
\end{itemize}
Entre las operaciones habilitadas para el tipo:
$$FALTA ESTO$$

\subsubsection{Arrays}
Todos los tipos pueden formar un array multidimensional, la sintaxis de declaración es la siguiente:
\begin{itemize}
    \item \textbf{Array}: \texttt{Tipo[DIMENSION] var;}
\end{itemize}
Entre las operaciones habilitadas para el tipo:
$$FALTA ESTO$$

\subsubsection{Funciones}
Las funciones se han declarado también como un tipo para poder hacer expresiones lambda y pasar funciones como argumento. La sintaxis de declaración de una función es la siguiente:
\begin{itemize}
    \item \textbf{Funciones}: \texttt{func var(Tipo arg1, Tipo arg2, ...) : TipoRetorno \{...\};}
\end{itemize}
El paso de parámetros por defecto es por valor, pero puede cambiarse a por referencia añadiendo el caracter ``\&'' al final del tipo del argumento.

\subsubsection{Punteros}
Pueden declarase punteros a cualquiera de los tipos definidos. La sintaxis de declaración es:
\begin{itemize}
    \item \textbf{Puntero\footnote{Como nota especial, la declaración de un puntero a estructuras de tipo array, se haría como \texttt{Tipo[DIMENSION]~ var;}.}}: \texttt{Tipo\~{} var}
\end{itemize}

\subsection{Tipos definidos por el usuario y constantes}
Adicionalmente, permitimos la definición de tipos por parte del usuario a través de la palabra reservada:
\begin{itemize}
    \item \textbf{Definición de tipos de usuario}: \texttt{taipdef nombre expresion}.
\end{itemize}
Por último, la declaración de constantes es posible gracias a la instrucción:
\begin{itemize}
    \item \textbf{Declaración de constantes}: \texttt{difain NOMBRE valor}
\end{itemize}

\subsection{Instrucciones del lenguaje}
El lenguaje tiene el siguiente repetorio de instrucciones:
\begin{itemize}
    \item \textbf{Instrucción de asignación}: \texttt{:=}
    \begin{center}
        \begin{minipage}{\linewidth}
            \begin{lstlisting}[linewidth=0.3\linewidth, gobble=16]
                int var := 3
            \end{lstlisting}
        \end{minipage}
    \end{center}
    
    \item \textbf{Instrucciones condicionales}: \texttt{if-else}, \texttt{switch}
    \begin{center}
        \begin{minipage}{\linewidth}
            \begin{lstlisting}[linewidth=0.3\linewidth, gobble=16]
                if (var > 3) {
                    ...
                }
                else {
                    ...
                }
            \end{lstlisting}
        \end{minipage}
    \end{center}

    \item \textbf{Instrucción de bucle}: \texttt{while}
    \begin{center}
        \begin{minipage}{\linewidth}
            \begin{lstlisting}[linewidth=0.3\linewidth, gobble=16]
                guail (var > 0) {
                    ...
                }
            \end{lstlisting}
        \end{minipage}
    \end{center}
    Se incluyen además las instrucciones \texttt{breic} y \texttt{continiu}.
    \item \textbf{Acceso a punteros}: \texttt{\~{}punt}
\end{itemize}


\subsection{Sucio}
\begin{itemize}
    \item \sout{Las constantes se declaran como los \texttt{\# define} de C++, pero con \texttt{\# difain}.}
    \item \sout{Los nombres de variable siguen los mismos patrones que en C++.}
    \item \sout{Los arrays como las variables normales, pero con corchetes indicando la dimensión (como en C++).}
    \item \sout{Indicamos que una variable es puntero con \texttt{int\~{}  var}.}
    \item \sout{Paso por valor y por referencia igual que en C++.}
    \item \sout{Los bloques tienen como separadores las llaves.}
    \item \sout{Los struct se escriben \texttt{estrut}}.
    \item \sout{El identificador de tipo es \texttt{clas}}.
    \item \sout{El identificador de los tipo entero es \texttt{int} y el de los booleanos es \texttt{bul}}.
    \item Operadores infijos:
    \begin{itemize}
        \item Aritméticas como C++, añadimos el operador exponencial.
        \item Asignación es \texttt{:=}.
        \item Igualdad es \texttt{=} y la desigualdad como en C++.
        \item Los operadores booleanos son: \texttt{an}, \texttt{or}, \texttt{sor}, \texttt{not} y menor mayor es como en C++.
        \item Operador parentesis y corchetes también.
    \end{itemize}
    \item \sout{Para definir los tipos \texttt{taipdef}.}
    \item \sout{Para definir las funciones \texttt{func nombre(args): tipoRetorno \{$\cdots$\}}.}
    \item Las funciones de entrada salida son \texttt{cein} y \texttt{ceaut} con las funciones de modificación de entrada salida que vayamos necesitando. La intrucción de retorno para las funciones es \texttt{return valor}
    \item \sout{Acceso a punteros \texttt{\~{}puntero}}
    \item \sout{Instrucciones del lenguaje:}
    \begin{itemize}
        \item \sout{Asignación \texttt{:=}}
        \item \sout{Condicionales \texttt{if () \{...\} else \{...\}} y \texttt{suich() \{ queis()...\}}}
        \item \sout{Bucle indefinido \texttt{guail}}
        \item Bucles definido \texttt{for}.
        \item \sout{Sentencias de control de bucles: \texttt{breic} y \texttt{continiu}.}
    \end{itemize}
\end{itemize}

\newpage
\appendix

\section{Ejemplos de programas habituales}

\end{document}
