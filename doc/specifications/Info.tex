\documentclass[10pt,a4paper]{article}

%% Formateo del título del documento
\title{\Huge C$+${\normalsize o}$-$}
\author{Juan Diego Barrado Daganzo, Javier Saras González y Daniel González Arbelo \\ 4º de Carrera}
\date{\today\footnote{Este documento se actualiza, para consultar las últimas versiones entrar en el enlace \url{https://github.com/JuanDiegoBarrado/PracticaPL}}}

%% Formateo del estilo de escritura y de la pagina
\pagestyle{plain}               % Estilo de página
\setlength{\parskip}{0.35cm}    % Edicion de espaciado
\setlength{\parindent}{0cm}     % Edicion de sangría
\clubpenalty=10000              % Llíneas viudas NO
\widowpenalty=10000             % Líneas viudas NO

%% Para establecer las medidas de los margenes
\usepackage[top=2.5cm, bottom=2.5cm, left=3cm, right=3cm]{geometry} 
%% Para que el idioma por defecto sea español
\usepackage[spanish]{babel}
%% Para poder subrayar entornos especiales como las secciones
\usepackage{ulem}

%% Texto matematico y simbolos especiales
\usepackage{amsmath}    % Paquete para mates
\usepackage{amsfonts}   % Paquete para mates
\usepackage{amssymb}    % Paquete para mates
\usepackage{stmaryrd}   % Paquete para mates
\usepackage{latexsym}   % Paquete para mates

%% Paquete para incluir imágenes y ruta de la carpeta de las imágenes
\usepackage{graphicx}
\graphicspath{{./fotos/}}

%% Paquete para tener hipervínculos y referencias cruzadas
\usepackage[colorlinks=true]{hyperref}
\hypersetup{
	urlcolor=red,
	linkcolor=blue,
}

%% Paquete para incluir código con coloreado sintáctico
\usepackage{listings}
\lstdefinelanguage{C+-}
{
  keywords={
    int, bul,
    if, else, guail, breic, continue,
    func, return
    },
  keywordstyle=\color{blue},
  commentstyle=\color{green},
  stringstyle=\color{red},
  sensitive=true,
  morecomment=[l]{//},
}

\lstset{
    language=C+-,
    basicstyle=\ttfamily\small,
    keywordstyle=\color{blue},
    commentstyle=\color{green},
    stringstyle=\color{red},
    numbers=left,
    numberstyle=\tiny\color{gray},
    breaklines=true,
    frame=single,
    backgroundcolor=\color{white},
    tabsize=4
}


%% Definicion de operadores especiales para simplificar la escritura matematica
\DeclareMathOperator{\dom}{dom}
\DeclareMathOperator{\img}{img}
\DeclareMathOperator{\rot}{rot}
\DeclareMathOperator{\divg}{div}
\newcommand{\dif}[1]{\ d#1}

%% Paquete e instrucciones para la generacion de los dibujos
\usepackage{pgfplots}
\pgfplotsset{compat=1.17}
\usepackage{tkz-fct}
\usepackage{pstricks}
\usepackage{pstcol} 
\usepackage{pst-node}
\usepackage{pst-plot}

%% Paquetes extra
\usepackage{centernot}  % Paquete para tachar cosas
\usepackage{appendix}   % Paquete para apéndice
\usepackage{verbatim}   % Paquete para comentar bloques de código de LaTeX

\begin{document}
\maketitle

\section{Especificaciones técnicas del lenguaje}
\subsection{Identificadores y ámbitos de definición}

\subsection{Tipos}
Los tipos básicos del lenguaje son los enteros y los booleanos (caracteres?), pudiendo formar arrays de estos tipos. Cada uno se declarará usando estas palabras reservadas:
\begin{itemize}
    \item \textbf{Enteros}: \texttt{int}
    \item \textbf{Booleanos}: \texttt{bul}
    \item \textbf{Array}: \texttt{Tipo[DIMENSION] var}
\end{itemize}
El lenguaje también posee otros tipos compuestos como clases y registros. Adicionalmente, se ha considerado un tipo para las funciones y para el vacío\footnote{El tipo vacío es necesario para poder }. Las palabras reservadas para ello son las siguientes:
\begin{itemize}
    \item \textbf{Clases}: \texttt{clas}
    \item \textbf{Registros}: \texttt{estrut}
    \item \textbf{Funciones}: \texttt{func}
    \item \textbf{Tipo vacío}: \texttt{void}
\end{itemize}
Por último, los punteros a cualquiera de estos tipos se declararán poniendo el caracter '\~{}' al final de la palabra reservada para el tipo.
\begin{itemize}
    \item \textbf{Puntero}: \texttt{Tipo\~{} var}
\end{itemize}
Como nota especial, la declaración de un puntero a estructuras de tipo array, se haría como \texttt{Tipo[DIMENSION]~ var}.

\subsection{Instrucciones del lenguaje}
El lenguaje tiene el siguiente repetorio de instrucciones:
\begin{itemize}
    \item \textbf{Instrucción de asignación}: \texttt{:=}
    \begin{lstlisting}
        int var := 3
    \end{lstlisting}
    \item \textbf{Instrucciones condicionales}: \texttt{if-else}, \texttt{switch}
    \begin{lstlisting}
        if (var > 3) {
            ...
        }
        else {
            ...
        }
    \end{lstlisting}
    \item \textbf{Instrucción de bucle}: \texttt{while}
    \begin{lstlisting}
        while (var > 0) {
            ...
        }
    \end{lstlisting}
    Se incluyen además las instrucciones \texttt{breic} y \texttt{continiu}.
\end{itemize}


\subsection{Sucio}
\begin{itemize}
    \item Las constantes se declaran como los \texttt{\# define} de C++, pero con \texttt{\# difain}.
    \item Los nombres de variable siguen los mismos patrones que en C++.
    \item \sout{Los arrays como las variables normales, pero con corchetes indicando la dimensión (como en C++).}
    \item Indicamos que una variable es puntero con \texttt{int\~{}  var}.
    \item Paso por valor y por referencia igual que en C++.
    \item \sout{Los bloques tienen como separadores las llaves.}
    \item \sout{Los struct se escriben \texttt{estrut}}.
    \item \sout{El identificador de tipo es \texttt{clas}}.
    \item \sout{El identificador de los tipo entero es \texttt{int} y el de los booleanos es \texttt{bul}}.
    \item Operadores infijos:
    \begin{itemize}
        \item Aritméticas como C++, añadimos el operador exponencial.
        \item Asignación es \texttt{:=}.
        \item Igualdad es \texttt{=} y la desigualdad como en C++.
        \item Los operadores booleanos son: \texttt{an}, \texttt{or}, \texttt{sor}, \texttt{not} y menor mayor es como en C++.
        \item Operador parentesis y corchetes también.
    \end{itemize}
    \item Para definir los tipos \texttt{taipdef}.
    \item Para definir las funciones \texttt{func nombre(args): tipoRetorno \{$\cdots$\}}.
    \item Las funciones de entrada salida son \texttt{cein} y \texttt{ceaut} con las funciones de modificación de entrada salida que vayamos necesitando. La intrucción de retorno para las funciones es \texttt{return valor}
    \item Acceso a punteros \texttt{\~{}puntero}
    \item \sout{Instrucciones del lenguaje:}
    \begin{itemize}
        \item \sout{Asignación \texttt{:=}}
        \item \sout{Condicionales \texttt{if () \{...\} else \{...\}} y \texttt{suich() \{ queis()...\}}}
        \item \sout{Bucle indefinido \texttt{guail}}
        \item \sout{Bucles definido \texttt{for}.}
        \item \sout{Sentencias de control de bucles: \texttt{breic} y \texttt{continiu}.}
    \end{itemize}
\end{itemize}

\end{document}
